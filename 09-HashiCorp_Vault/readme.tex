\documentclass[12pt]{article}
\usepackage[margin=1in,headheight=30pt,footskip=37pt]{geometry}
\usepackage{fontspec}
\usepackage{longtable,booktabs}
\usepackage{hyperref}
\usepackage{listings}
\usepackage{xcolor}
\usepackage{fancyhdr}
\usepackage{titling}
\usepackage{array}
\usepackage{graphicx}

% Font Configuration - STC Forward with fallback
\IfFontExistsTF{STC Forward}{
  \setmainfont{STC Forward}
}{
  \setmainfont{Arial}
}

% Set monospace font with system fallbacks
\IfFontExistsTF{Monaco}{
  \setmonofont{Monaco}[Scale=0.85]
}{
  \IfFontExistsTF{Menlo}{
    \setmonofont{Menlo}[Scale=0.85]
  }{
    \setmonofont{Latin Modern Mono}[Scale=0.85]
  }
}


% Brand Colors - Solutions.com.sa
\definecolor{solutionsblue}{HTML}{0066CC}
\definecolor{solutionsgray}{HTML}{666666}
\definecolor{solutionslight}{HTML}{E6F2FF}

% Brand Identity Macro
\newcommand{\solutionsbrand}{\textcolor{solutionsblue}{\textbf{solutions.com.sa}}}

% Header/Footer Configuration
\pagestyle{fancy}
\fancyhf{}
\fancyhead[L]{\small\textcolor{solutionsgray}{HashiCorp Vault Automation}}
\fancyhead[R]{\small\textcolor{solutionsblue}{\textbf{solutions.com.sa}} \quad \includegraphics[width=2cm]{logo.pdf}}
\fancyfoot[L]{\small\textcolor{solutionsgray}{www.solutions.com.sa}}
\fancyfoot[C]{\small\textcolor{solutionsgray}{\thepage}}
\fancyfoot[R]{\small\textcolor{solutionsgray}{Infrastructure Solutions}}
\renewcommand{\headrulewidth}{0.5pt}
\renewcommand{\footrulewidth}{0.5pt}
\renewcommand{\headrule}{\hbox to\headwidth{\color{solutionsblue}\leaders\hrule height \headrulewidth\hfill}}
\renewcommand{\footrule}{\hbox to\headwidth{\color{solutionsblue}\leaders\hrule height \footrulewidth\hfill}}

\hypersetup{colorlinks=true,linkcolor=solutionsblue,urlcolor=solutionsblue,breaklinks=true}

\lstdefinelanguage{bash}{
  morekeywords={export,if,then,fi,for,do,done,while},
  sensitive=true
}
\lstset{
  basicstyle=\ttfamily\footnotesize,
  keywordstyle=\color{blue},
  commentstyle=\color{gray},
  showstringspaces=false,
  frame=single,
  columns=fullflexible,
  breaklines=true,
  breakatwhitespace=true
}

\title{\textcolor{solutionsblue}{\textbf{Enterprise HashiCorp Vault Automation}} \\
       \Large\textcolor{solutionsgray}{Podman Deployment \& Intelligent File Sync Solutions} \\
       \vspace{0.5cm}
       \large Powered by \solutionsbrand}
\author{\textcolor{solutionsgray}{Infrastructure Automation Division}}
\date{\textcolor{solutionsgray}{\today}}

% Title Page Styling
\pretitle{\begin{center}\color{solutionsblue}\LARGE\bfseries}
\posttitle{\par\end{center}\vspace{0.5cm}{\color{solutionsblue}\rule{\linewidth}{2pt}}\vspace{1cm}}

\begin{document}
\maketitle
\thispagestyle{fancy}

\section*{Abstract}
This document summarizes a single-node HashiCorp Vault deployment automated with Podman plus a helper script (\texttt{sync\_on\_change\_clean.sh}) that watches a local file, securely syncing changes to a remote host. The watcher now supports layered configuration precedence and per-variable source reporting.

\section*{Brand Alignment}
This automation solution reflects \solutionsbrand{}'s commitment to enterprise-grade infrastructure automation. Our approach emphasizes:
\begin{itemize}
  \item \textcolor{solutionsblue}{\textbf{Security First}}: TLS-enabled deployments with certificate management
  \item \textcolor{solutionsblue}{\textbf{Operational Excellence}}: Automated deployment with comprehensive health monitoring
  \item \textcolor{solutionsblue}{\textbf{Intelligent Automation}}: Smart file synchronization with configurable precedence
  \item \textcolor{solutionsblue}{\textbf{Enterprise Ready}}: Production-focused with secure defaults and validation
\end{itemize}

For additional infrastructure solutions and consulting services, visit\\
\href{https://www.solutions.com.sa}{\textcolor{solutionsblue}{\textbf{www.solutions.com.sa}}}.

\section{Overview}
\begin{itemize}
  \item Installs / verifies Podman (and optionally Vault CLI).
  \item Prepares directory structure under \texttt{\$HOME/vault-server}.
  \item Generates (or reuses) a self-signed TLS certificate.
  \item Optionally trusts the certificate system-wide.
  \item Applies secure filesystem permissions (with permissive fallback).
  \item Writes \texttt{config.hcl} using file storage backend + TLS listener.
  \item Optionally validates the configuration.
  \item Manages firewall openings (unless disabled).
  \item Starts or replaces the Vault container and polls health.
\end{itemize}

\section{Quick Start (Non-Interactive)}
\begin{lstlisting}[language=bash]
TRUST_CERT=1 INSTALL_CLI=1 ./prod/install_vault_container_prod-moduler-version-clean-working.sh
\end{lstlisting}
If the certificate is not trusted:
\begin{lstlisting}[language=bash]
export VAULT_ADDR=https://127.0.0.1:8200
export VAULT_SKIP_VERIFY=1   # Dev only; prefer --cacert
\end{lstlisting}

\section{Initialize \& Unseal}
\begin{lstlisting}[language=bash]
vault operator init
vault operator unseal
vault status
\end{lstlisting}

\section{Core Environment Variables}
\footnotesize
\begin{longtable}{@{}p{3.5cm}p{4.5cm}p{2.5cm}@{}}
\toprule
Variable & Purpose & Default \\
\midrule
\endhead
VAULT\_VERSION & Vault image tag & \texttt{latest} \\
VAULT\_PORT & API listen port & 8200 \\
VAULT\_CLUSTER\_PORT & Cluster port & 8201 \\
VAULT\_API\_ADDR & Advertised API & \texttt{https://127.0.0.1:8200} \\
INSTALL\_CLI & Install Vault CLI (1=yes) & 0 \\
TRUST\_CERT / TRUST\_VAULT\_CERT & Trust self-signed cert & 0 \\
PERMISSIVE\_STORAGE & Force wide-open perms & 0 \\
FIREWALL\_DISABLE & Skip firewall changes & 0 \\
CHECK\_VAULT\_CONFIG & Run \texttt{-check-config} & 1 \\
\bottomrule
\end{longtable}
\normalsize

\section{Directory Layout}
\begin{lstlisting}
$HOME/vault-server/
  data/
    certs/
      public.crt
      private.key
    storage/
  config/
    config.hcl
\end{lstlisting}

\section{Container Mounts}
\begin{verbatim}
/data   -> data + certs
/config -> configuration
\end{verbatim}

\section{Health \& Status}
\begin{lstlisting}[language=bash]
curl --cacert $HOME/vault-server/data/certs/public.crt https://127.0.0.1:8200/v1/sys/health
# Dev only:
curl -k https://127.0.0.1:8200/v1/sys/health
podman logs vault | head
podman exec vault vault status -tls-skip-verify
\end{lstlisting}
Return codes:
\begin{itemize}
  \item 501 = uninitialized
  \item 503 = sealed
\end{itemize}

\section{Updating Vault}
\begin{lstlisting}[language=bash]
VAULT_VERSION=1.20.2 ./prod/install_vault_container_prod-moduler-version-clean-working.sh
\end{lstlisting}
Omit \texttt{VAULT\_VERSION} to keep \texttt{latest}.

\section{Clean Removal}
Script:
\begin{lstlisting}[language=bash]
./clean_vault_script.sh          # interactive
./clean_vault_script.sh -f       # forced
CLEAN_IMAGE=1 CLEAN_CLI=1 ./clean_vault_script.sh -f
\end{lstlisting}
Manual:
\begin{lstlisting}[language=bash]
podman rm -f vault
rm -rf $HOME/vault-server
\end{lstlisting}

\section{Security Notes}
\begin{itemize}
  \item Replace self-signed cert in production.
  \item File storage backend is single-node; use Raft/external for HA.
  \item Avoid permissive permissions outside dev.
  \item Protect unseal keys and root token.
  \item Pin explicit Vault versions.
\end{itemize}

\section{Troubleshooting}
\footnotesize
\begin{longtable}{@{}p{3.5cm}p{7.5cm}@{}}
\toprule
Symptom & Action \\
\midrule
\endhead
Startup timeout & \texttt{podman logs vault} (check TLS, perms, mlock) \\
Certificate errors & Use \texttt{TRUST\_CERT=1} or \texttt{--cacert} \\
Port in use & \texttt{ss -tulnp | grep ':8200'} then free or override \\
501 health & Initialize (\texttt{vault operator init}) \\
503 health & Unseal (\texttt{vault operator unseal}) \\
\bottomrule
\end{longtable}
\normalsize

\section{Watcher: \texttt{sync\_on\_change\_clean.sh}}
Purpose: monitor a single local file, detect content changes (SHA-256), scp it to a remote host, and set execute permission remotely.

\subsection*{Usage}
\begin{lstlisting}[language=bash]
./sync_on_change_clean.sh <file> [-c <config-file>] [-h]
\end{lstlisting}

\subsection*{Examples}
\begin{lstlisting}[language=bash]
./sync_on_change_clean.sh prod/install_vault_container_prod-moduler-version-clean-working.sh
./sync_on_change_clean.sh prod/install_vault_container_prod-moduler-version-clean-working.sh -c ./sync_on_change.conf
\end{lstlisting}

\subsection*{Configuration Precedence (first found wins)}
\begin{enumerate}
  \item \texttt{-c <config-file>} (explicit; must exist; no fallback)
  \item \texttt{<script\_dir>/sync\_on\_change.conf}
  \item \texttt{\$PWD/.sync\_on\_change.conf}
  \item \texttt{<script\_dir>/.sync\_on\_change.conf}
  \item \texttt{<watched\_file\_dir>/.sync\_on\_change.conf}
\end{enumerate}

\subsection*{Overridable Variables}
\begin{itemize}
  \item \texttt{remote\_user}
  \item \texttt{remote\_host}
  \item \texttt{remote\_path}
  \item \texttt{interval}
  \item \texttt{max\_failures}
\end{itemize}

\subsection*{Startup Reporting}
The script prints each variable along with its source (default vs the config file path).

\subsection*{Sample Configuration File}
\begin{lstlisting}[language=bash]
# sync_on_change.conf
remote_user="username"
remote_host="172.00.00.00"
remote_path="~/vault-scripts/"
interval=2
max_failures=15
\end{lstlisting}

\section{Minimal Workflow}
\begin{verbatim}
Run install script -> vault operator init -> vault operator unseal
Set VAULT_ADDR -> use Vault
\end{verbatim}

\section{Production Reminders}
Harden TLS, use HA backend, secure unseal keys, pin versions, monitor logs.

\end{document}
